\section{Event reconstruction}
\label{sec:Objects}

In the CMS detector, the Particle-Flow (PF) algorithm~\cite{Sirunyan:2017ulk} is used to reconstruct and identify each particle.
Using the combination of all sub-detector information, individual particles are defined as charged and neutral hadrons, leptons, and photons.
The missing transverse momentum $p_T^{miss}$ is defined as the magnitude of the negative vector sum of the transverse momentum of all the reconstructed particles in the event.

For each event, the primary vertex (PV), defined as the vertex with the highest sum of $p_{T}^{2}$ from charged tracks, is chosen. 

Photons are reconstructed using a multivariate technique based on a Boosted Decision Tree (BDT) trained to separate photons from jets~\cite{Sirunyan:2018ouh}.
Corrections are applied to fix the mis-modelling of radiation lost in material upstream in the ECAL
and imperfect shower contamination of MC samples using the Chained Quantile Regression (CQR) method as described in~\cite{Khachatryan:2014ira}.
In all three decay channels of WW (in $pp \rightarrow HH \rightarrow WW\gamma\gamma$), a common photon pre-selection is applied.
Each channel requires a diphoton candidate, which is defined as a pair of two photons.
The leading photon is required to have $p_T > 35 ~GeV$ and the sub-leading photon is required to have $p_T > 25 ~GeV$.
Both photons are required to be within the ECAL and tracker region ($|\eta| < 2.5$). Along with this,
the invariant mass of the diphoton system is required to be between 100 GeV and 180 GeV.
If more than one diphoton candidate is present in an event, the one with the highest diphoton $p_T$ is chosen.

The leptons considered are required to be associated with the primary vertex of the event~\cite{Chatrchyan:2014fea}
to suppress electron candidates originating from photon conversion, and lepton candidates originating
from in-flight decays of heavy quarks. Additionally, leptons are required to be isolated from the other particles in the event.
Relative isolation is defined in Eq. \ref{eq:isolation}, where the summation is over all of the charged and neutral hadrons and photons
in a cone defined by $\Delta R = \sqrt{(\Delta \eta)^2 + (\Delta \phi)^2} = 0.3 (0.4)$ around the trajectory of the electron (muon), 
and $p_T^{\textnormal{PU}}$ denotes the contribution of neutral particles from pileup~\cite{Khachatryan:2015hwa,Sirunyan:2018fpa}.
 
\begin{equation} \label{eq:isolation}
    R_{iso} = \bigg[\sum_{\substack{charged\\hadrons}}p_T+max \big(0, \sum_{\substack{neutral\\hadrons}}p_T + \sum_{photons}p_T - p_T^{\textnormal{PU}} \big)\bigg]/p_T^{l}
\end{equation}

This analysis selects loose electrons, and tight muons. Loose electrons are defined as PF electrons which pass a loose cut based ID, which includes a loose isolation selection. Tight muons are defined 
as PF muons which pass a tight identification score requirement. Loose electrons (tight muons) have an average efficiency of $\approx$ 90\% (90\%), measured using
a high-purity Drell-Yan MC sample. 

In addition to a loose electron ID, electron candidates are required to have $p_{T} > $ 10 GeV, and a pseudorapidity in the range (0 $<$ $|\eta|$ $<$ 1.4442) or (1.566 $<$ $|\eta|$ $<$ 2.5) 
in order to remain in the CMS tracker region and avoid the ECAL overlap region. Furthermore, a distance parameter value ($\Delta R = \sqrt{\Delta\eta^{2} + \Delta\phi^{2}}$) greater than 
0.4 is required between each electron candidate and each of the two photon candidates from the event's highest $p_{T}$ diphoton in order to select isolated electron candidates. A distance parameter value 
of less than 0.4 is also required between the electron candidate's track and ECAL supercluster position, and a 
distance parameter value with each jet candidate $>$ 0.4 is required. Finally, the invariant mass of the electron with each photon candidate in the event's highest \pt diphoton 
candidate must be at least 5 GeV greater or less than the Z boson mass in order to avoid selecting events coming from Z$\rightarrow$ee decays. 

In addition to a tight Muon ID, muon candidates are required to have $p_{T} > $ 10 GeV, a pseudorapidity in the range ($|\eta| < 2.4$) to remain in the CMS tracker region, a distance parameter value with each photon candidate $>$ 0.4, a 
distance parameter value with each jet candidate $>$ 0.4, and an isolation $<$ 0.15, as defined in Eq. \ref{eq:isolation}, in order to select isolated muon candidates. 

Jets are constructed using the anti-$k_{T}$ clustering method, classifying them as AK4 jets with a distance parameter of 0.4. Jet candidates are 
required to have $p_{T} > $ 25 GeV, an absolute value of pseudorapidity $<$ 2.4, are required to pass a loose PU Jet ID in order to avoid reconstructing
jets from pileup interactions, a distance parameter value $>$ 0.4 between the jet 
candidate and each photon candidate from the diphoton candidate, and a distance parameter $>$ 0.4 with any electron and muon candidates which pass the previously defined 
electron and muon selections. In addition, jets from the hadronization of bottom quarks are tagged using a Deep Neural Network (DNN) that takes secondary vertices and PF 
candidates as inputs \cite{Sirunyan:2017ezt}. The output of this DNN is referred to as the b-tagging score.