\section{The CMS detector}\label{section:CMS_Detector}

A detailed description of the CMS detector, together with a definition of the coordinate system used and the relevant
kinematic variables, can be found in Ref.~\cite{Chatrchyan:2008zzk}.

The central feature of the CMS apparatus is a superconducting solenoid of 6 \unit{m} internal diameter, providing a
magnetic field of 3.8 \unit{T}. Within the solenoid volume are a silicon pixel and strip tracker, a lead tungstate
crystal electromagnetic calorimeter (ECAL), and a brass and scintillator hadron calorimeter (HCAL), each composed of a
barrel and two endcap sections. Forward calorimeters extend the pseudorapidity coverage provided by the barrel and
endcap detectors. Muons are measured in gaseous detectors embedded in the steel flux-return yoke outside the solenoid.
The particle-flow algorithm~\cite{CMS-PRF-14-001} reconstructs and identifies each individual particle in an event, with
an optimized combination of information from the various elements of the CMS detector.  The energy of photons is
obtained from the ECAL measurement.  The energy of electrons is determined from a combination of the electron momentum
at the primary interaction vertex as measured by the tracker, the energy of the corresponding ECAL cluster, and the
energy sum of all bremsstrahlung photons spatially compatible with originating from the electron track.  The energy of
muons is obtained from the curvature of the corresponding track.  The energy of charged hadrons is determined from a
combination of their momentum measured in the tracker and the matching ECAL and HCAL energy deposits, corrected for the
response function of the calorimeters to hadronic showers.  Finally, the energy of neutral hadrons is obtained from the
corresponding corrected ECAL and HCAL energies.

Events of interest are selected using a two-tiered trigger system.  The first level, composed of custom hardware
processors, uses information from the calorimeters and muon detectors to select events at a rate of up to 100\unit{kHz}
within a fixed latency of about 4\mus~\cite{Sirunyan:2020zal}.  The second level, known as the high-level trigger,
consists of a farm of processors running a version of the full event reconstruction software optimized for fast
processing, and reduces the event rate to around 1\unit{kHz} before data storage~\cite{Khachatryan:2016bia}.


% All of the answers are here: https://twiki.cern.ch/twiki/bin/viewauth/CMS/Internal/PubDetector

% \chapter{The CMS detector} \label{section:CMS_Detector}



% Events of interest are selected using a two-tiered trigger system. The first level (L1), composed of custom hardware processors, uses information from the 
% calorimeters and muon detectors to select events at a rate of around 100\unit{kHz} within a fixed latency of about 4\mus~\cite{CMS:2020cmk}. The second level, 
% known as the high-level trigger (HLT), consists of a farm of processors running a version of the full event reconstruction software optimized for fast processing, 
% and reduces the event rate to around 1\unit{kHz} before data storage~\cite{CMS:2016ngn}.

% for now just including a bit on photons since they are the most key object in the analysis since their invariant mass defines our signal region - maybe consider adding 
% additional object info since this analysis uses many objects. 

The electromagnetic calorimeter consists of 75\,848 lead tungstate crystals, which provide coverage in pseudorapidity $\abs{\eta} < 1.48 $ in a barrel region (EB) 
and $1.48 < \abs{\eta} < 3.0$ in two endcap regions (EE). Preshower detectors consisting of two planes of silicon sensors interleaved with a total of $3 X_0$ of 
lead are located in front of each EE detector.

In the barrel section of the ECAL, an energy resolution of about 1\% is achieved for unconverted or late-converting photons in the tens of GeV energy range. 
The energy resolution of the remaining barrel photons is about 1.3\% up to $\abs{\eta} = 1$, changing to about 2.5\% at $\abs{\eta} = 1.4$. In the endcaps, 
the energy resolution is about 2.5\% for unconverted or late-converting photons, and between 3 and 4\% for the other ones~\cite{CMS:2015myp}.

The diphoton mass resolution, as measured in \Htogg decays, is typically in the 1--2\% range, depending on the measurement of the photon 
energies in the ECAL and the topology of the photons in the event~\cite{CMS:2020xrn}.

The integrated luminosities for the 2016, 2017, and 2018 data-taking years have 1.2--2.5\% individual uncertainties~\cite{CMS-LUM-17-003,CMS-PAS-LUM-17-004,CMS-PAS-LUM-18-002}, 
while the overall uncertainty for the 2016--2018 period is 1.6\%.